%%%%%%%%%%%%%%%%%%%%%%%%%%%%%%%%%%%%%%%%%
% Medium Length Professional CV
% LaTeX Template
% Version 2.0 (8/5/13)
%
% This template has been downloaded from:
% http://www.LaTeXTemplates.com
%
% Original author:
% Trey Hunner (http://www.treyhunner.com/)
%
% Important note:
% This template requires the resume.cls file to be in the same directory as the
% .tex file. The resume.cls file provides the resume style used for structuring the
% document.
%
%%%%%%%%%%%%%%%%%%%%%%%%%%%%%%%%%%%%%%%%%

%----------------------------------------------------------------------------------------
%	PACKAGES AND OTHER DOCUMENT CONFIGURATIONS
%----------------------------------------------------------------------------------------
\pdfmapfile{=raleway.map}
\documentclass{resume_anzy} % Use the custom resume.cls style

%\usepackage[default]{opensans}
\usepackage[T1]{fontenc}
\usepackage[default]{raleway}
\usepackage{hyperref}
\usepackage{xcolor}

\usepackage[left=0.75in,top=0.6in,right=0.75in,bottom=0.6in]{geometry} % Document margins

\name{Anzy Lee} % Your name

\address{Department of Civil and Environmental Engineering, Utah State University \\ Logan, UT 84322} % Your address
\address{anzy.lee@usu.edu \\ \href{https://anzylee.github.io}{\textcolor{blue}{https://anzylee.github.io}}} % Your phone number and email

\address{\bf \large{Research Scientist}}


\begin{document}

%----------------------------------------------------------------------------------------
%	Research interests
%----------------------------------------------------------------------------------------
%\vspace{5mm}
%\begin{rSection}{Research Interests}
%\begin{rSubsection}{}{}{}{}
%\vspace{-2.5mm}
%\item {\bf Flow and transport problems:} Hydrodynamic models, CFD, Two-phase flow, Flow in porous media, Coupled systems
%\item {\bf Machine Learning:} Neural Networks, Metaheuristic Optimization Algorithm
%\end{rSubsection}
%\end{rSection}

%----------------------------------------------------------------------------------------
%	EDUCATION SECTION
%----------------------------------------------------------------------------------------

I am a river scientist specializing in environmental hydrodynamics, fluvial geomorphology, and watershed hydrology, with strong expertise in hydrologic and hydraulic modeling, geospatial analysis, and statistical analysis. My mission is to understand watershed dynamics and nteractions to enhance ecosystem function and resilience.



\begin{rSection}{Education}

{\bf Ph.D in Civil Engineering} \hfill {\em 2016 - 2020} \\ 
\textit{Purdue University}  
%\\ \textit{Riverbed Morphology, Hydrodynamics and Hyporheic Exchange Processes}

{\bf MS in Civil and Environmental Engineering } \hfill {\em 2014 - 2016} \\ 
\textit{Seoul National University, Republic of Korea}

{\bf BS in Construction, Urban and Environmental Engineering} \hfill {\em 2010 - 2014} \\ 
\textit{Handong Global University, Republic of Korea}
% \\
%Thesis: Derivation of the Formation Mechanism of Beach Cusp \\
%Advisor: Prof. Kyungmo Ahn
\end{rSection}

%----------------------------------------------------------------------------------------
%	RESEARCH EXPERIENCE SECTION
%----------------------------------------------------------------------------------------

\begin{rSection}{Work Experience}

%------------------------------------------------


{\bf Research Scientist} \hfill {\em 2020 - Current} \\ 
\textit{Utah State University, UC Davis}

{\bf Research/Teaching Assistant} \hfill {\em 2016 - 2020} \\ 
\textit{Purdue University}

%
%
%%------------------------------------------------
%\vspace{-2.5mm}
%\begin{rSubsection}{Visiting Scholar}{Aug 2020 - Current}{Dr. Greg Pasternack}{\textit{Land, Air, and Water Resources, University of California, Davis}}
%\item  Synthesized a river archetype model representing various geomorphological features observed in natural riverine systems
%\end{rSubsection}
%
%%------------------------------------------------
%\vspace{-2.5mm}
%\begin{rSubsection}{Research Assistant}{Aug 2016 - Jul 2020}{Dr. Antoine Aubeneau}{\textit{Lyles School of Civil Engineering, Purdue University}}
%\item Numerical modeling of hyporheic exchange processes in fractal riverbed
%\end{rSubsection}
%

%------------------------------------------------
%\vspace{-2.5mm}
%\begin{rSubsection}{Visiting Scholar}{Feb 2019 - Apr 2019}{Prof. Xiaofeng Liu}{\textit{Civil and %Environmental Engineering, Penn State University}}
%\item Investigated boulder-driven hyporheic exchange 
%\end{rSubsection}

%------------------------------------------------
%\vspace{-2.5mm}
%\begin{rSubsection}{Visiting Scholar}{Jan 2018 - Jan 2019}{Prof. M. Bayani Cardenas}{\textit{Jackson School of Geosciences, The University of Texas at Austin}}
%\item  Studied hyporheic exchange in channels with high Froude Number flows: the importance of free %surface water elevation changes
%\end{rSubsection}

%------------------------------------------------
%\vspace{-2.5mm}
%\begin{rSubsection}{Research Assistant}{2014 - 2015}{Prof. Kyung-Duck Suh}{\textit{Coastal Engineering Laboratory, Seoul National University}}
%\item Developed a robust hybrid Artificial Neural Network (ANN) model integrated with the Harmony search algorithm to estimate the stability number of armor unit of rubble mound structure 
%\end{rSubsection}


%------------------------------------------------

\end{rSection}


%\begin{rSection}{AWARDS AND EXTRACURRICULAR  EXPERIENCE}
%\vspace{-2.5mm}
%\item \textbf{Dorothy Faye Dunn Fellowship}, \textit{Purdue University} \hfill{2019}
%\item \textbf{Climate Science Summer School}, \textit{NASA JPL Center for Climate Sciences} \hfill{2018}
%\item \textbf{Delleur Award}, \textit{Purdue University} \hfill{2017, 2018}
%\item \textbf{Summer Institute on Earth-Surface Dynamics}, \textit{National Center for Earth-surface Dynamics} \hfill{2017}
%%\item \textbf{LG Foundation Fellowships}, \textit{LG Yonam Foundation} \hfill{2014-2015}
%%\item \textbf{Student Representative}, Spatial Environment System Engineering, \textit{Handong University} \hfill{2012}
%%\item \textbf{Merit Scholarship (Top 5$\%$)}, Spatial Environment System Engineering, \textit{Handong University} \hfill{2012}
%\end{rSection}
%
%
%\vspace{2.5mm}

\begin{rSection}{RESEARCH PROJECTS}


{\bf National Oceanic and Atmospheric Association [\$1,500,000]} \hfill {\em 2023 - Current} \\  \textit{Novel Geospatial Architecture of Channel and Floodplain Morphological Attributes within the OWP Hydrofabrics}

{\bf California State Water Resources Board [\$3,000,000]} \hfill {\em 2020 - Current} \\ \textit{Application of methods and models to support the development and implementation of policies for water quality control for cannabis cultivation}


{\bf Purdue Research Foundation} \hfill {\em 2016 - 2020}


\end{rSection}


\begin{rSection}{SELECTED PEER-REVIEWED JOURNAL ARTICLES}
\vspace{-2.5mm}


\item \textbf{Lee, A.}, Castejon, J., Patterson, N., Diehl, R., Phillips, C. B., and Lane, B. \textit{(In Preparation)} Probabilistic quantification of within-reach hydraulic geometry variability. 

\item Castejon, J., \textbf{Lee, A.}, Patterson, N. K., Lane, B., and Phillips, C. B. \textit{(Submitted)} Leveraging High-resolution Topography to Advance Parsimonious Reach-scale Flood Inundation Mapping. 

\item \textbf{Lee, A.}, Lane, B., and Pasternack, G. B. \textit{(Submitted)} RiverSTICH: A modular synthetic 3D channel terrain generator from sparse transect data. 

\item \textbf{Lee, A.}, Lane, B., and Pasternack, G. B. (2025). Spectral Slope and Coherence Quantitatively Summarize Nested Topographic Variability Patterns in Rivers. \textit{River Res. Applic.}, 41: 1093-1103. \\ \href{https://doi.org/10.1002/rra.4437}{\textcolor{blue}{\textcolor{blue}{doi:10.1002/rra.4437}}} 

\item \textbf{Lee, A.}, Lane, B., and Pasternack, G. B. (2023). Identifying key channel variability functions controlling ecohydraulic conditions using synthetic channel archetypes.  \textit{Ecohydrology}, e2533. \\ \href{https://doi.org/10.1002/eco.2533}{\textcolor{blue}{doi:10.1002/eco.2533}}

\item \textbf{Lee, A.}, Cardenas, M. B., Aubeneau, A., and Liu, X. (2022). Hyporheic exchange due to cobbles on sandy beds. \textit{Water Resour. Res.} 58, e2021WR030164. \href{https://doi.org/10.1029/2021WR030164}{\textcolor{blue}{doi:10.1029/2021WR030164}}

\item \textbf{Lee, A.}, Cardenas, M. B., Aubeneau, A., and Liu, X. (2021). Hyporheic Exchange in Sand Dunes Under a Freely Deforming River Water Surface. \textit{Water Resour. Res.} 57, e2020WR028817. \\ \href{https://doi.org/10.1029/2020WR028817}{\textcolor{blue}{doi:10.1029/2020WR028817}}

\item \textbf{Lee, A.}, Cardenas, M. B., and Aubeneau, A.  (2020). The Sensitivity of Hyporheic Exchange to Fractal Properties of Riverbeds. \textit{Water Resour. Res.} 56, e2019WR026560. \href{https://doi.org/10.1029/2019WR026560}{\textcolor{blue}{doi:10.1029/2019WR026560}}

%\item Kim, S. W., \textbf{Lee, A.}, and Mun, J. (2018). A Surrogate Modeling for Storm Surge Prediction Using an Artificial Neural Network. \textit{J. of Coastal Res.} 84, 866-870. \href{https://doi.org/10.2112/SI85-174.1}{\textcolor{blue}{doi:10.2112/SI85-174.1}}

\item \textbf{Lee, A.}, Geem, J. W., and Suh, K. D. (2016). Determination of near-global optimal initial weights of artificial neural network using harmony search algorithm: Application to breakwater armor stones. \textit{Appl. Sci.} 6(6), 164. \href{https://doi.org/10.3390/app6060164}{\textcolor{blue}{doi:10.3390/app6060164}}

\item \textbf{Lee, A.}, Kim, S. E., and Suh, K. D. (2016). An easy way to use artificial neural network model for calculating stability number of rock armor. \textit{Ocean Eng.} 127, 349-356.  \href{https://doi.org/10.1016/j.oceaneng.2016.10.013}{\textcolor{blue}{doi:10.1016/j.oceaneng.2016.10.013}}
\end{rSection}

%----------------------------------------------------------------------------------------
%	AWARD & SERVICE
%----------------------------------------------------------------------------------------


\begin{rSection}{AWARD \& SERVICE}
\vspace{-2.5mm}
\item \textbf{Peer Reviewer} \hfill{2019 - Current} \\
\textit{River Res. Applic., Earth Surf. Process. Landf., Water Resour. Res., J. Hydrol., J. Hydraul. Eng.}

\item \textbf{Dorothy Faye Dunn Fellowship}, \textit{Purdue University} \hfill{2019}

%\item \textbf{Climate Science Summer School}, \textit{NASA JPL Center for Climate Sciences} \hfill{2018}

\item \textbf{Delleur Award}, \textit{Purdue University} \hfill{2017, 2018}

%\item \textbf{Summer Institute on Earth-Surface Dynamics}, \textit{National Center for Earth-surface Dynamics} \hfill{2017}
%\item \textbf{LG Foundation Fellowships}, \textit{LG Yonam Foundation} \hfill{2014-2015}
%\item \textbf{Student Representative}, Spatial Environment System Engineering, \textit{Handong University} \hfill{2012}

\item \textbf{Merit Scholarship (Top 5$\%$)}, \textit{Handong University} \hfill{2012}
\end{rSection}
%----------------------------------------------------------------------------------------
%	COMPUTER SECTION
%----------------------------------------------------------------------------------------


%\begin{rSection}{COMPUTER SKILLS }
%\vspace{-2.5mm}
%\item \textbf{Operating Systems}: Windows, Linux
%\item \textbf{Programming}: C/C++, MATLAB, Python, MPI, Visual Basic
%\item \textbf{Scientific Applications}: \LaTeX, OpenFOAM, FEniCS, ParaView, GIS, HEC-RAS, HEC-HMS  
%\item \textbf{Technical Drawing}: Inkscape, Adobe Illustrator, AutoCAD, Microsoft Visio
%\end{rSection}
%%
%\newpage
%----------------------------------------------------------------------------------------
%	TECHNICAL STRENGTHS SECTION
%----------------------------------------------------------------------------------------


\begin{rSection}{SELECTED CONFERENCE PROCEEDINGS}
\vspace{-2.5mm}
%\begin{rSection}{CONFERENCES AND SEMINARS}

\item Stieve, J., Lane, B., and \textbf{Lee, A.} \textit{(Submitted to ISE 2026).} Generating Syntetic Terrain Models from Sparse Historic Datasets for Ecohydraulic Assessment. 

\item \textbf{Lee, A.}, Castejon, J., Patterson, N., Diehl, R., Phillips, C. B., and Lane, B. \textit{(Accepted to AGU 2025. Oral Presentation).} Probabilistic quantification of within-reach hydraulic geometry. 

\item Castejon Villalobos, J., \textbf{Lee, A.,}, Lane, B., and Phillips, C. B.  (2024). Accurate River Channel Representation Within a HAND-y Method for Flood Inundation Mapping. \textit{AGU 2024 Fall Meeting.} Dec 2024, Washington D.C., United States.

%\item Patterson, N., Castejon Villalobos, J., \textbf{Lee, A.}, Lane, B., Diehl, R. M., and Phillips, C. B. (2024). Leveraging High-Resolution Topography to Determine the Bankfull Channel. AGU 2024 Fall Meeting, Dec 2024, Washington D.C., United States.

%\item \textbf{Lee, A.}, Lane, B., and Pasternack, G. B. (2022). Developing Archetypal River Corridor Terrain Models for Various Channel Types. AGU 2022 Fall Meeting, Dec 2022, Chicago, United States.

\item \textbf{Lee, A.}, Lane, B., Pasternack, G. B., and Sandoval-Solis, S. (2021). Identifying key geomorphic parameters characterizing eco-hydraulic responses of river channels using RiverBuilder. \textit{AGU 2021 Fall Meeting.} Dec 2021, New Orleans, United States.

%\item \textbf{Lee, A.}, Cardenas, M. B., and Aubeneau, A. (2018). Investigation of hyporheic exchange in channels with high Froude Number flows: the importance of free surface water elevation changes, AGU 2018 Fall Meeting, Dec 2018, Washington, D.C., United States. 

%\item Aubeneau, A. and \textbf{Lee, A.} (2018). Aris method for (reactive) transient storage models, AGU 2018 Fall Meeting, Dec 2018, Washington, D.C., United States.

\item \textbf{Lee, A.} and Aubeneau, A. (2017). 3D Numerical Modeling of Hyporheic Exchange Processes in Fractal Riverbed. \textit{AGU 2017 Fall Meeting.} Dec 2017, New Orleans, United States.

\end{rSection}

%\newpage


%----------------------------------------------------------------------------------------
%	DIGITAL PRODUCTS
%----------------------------------------------------------------------------------------


\begin{rSection}{DIGITAL PRODUCTS}
\vspace{-2.5mm}

\item \textbf{Lee, A.} and Lane, B. (2024). HAND-FIM Assessment Tools [Python]. GitHub. \href{https://github.com/USU-CIROH/HAND-FIM_Assessment_public}{\textcolor{blue}{Link.}}

\item \textbf{Lee, A.} (2024). Width Extraction Tools [Python]. GitHub. \href{https://github.com/USU-CIROH/width_extraction_public}{\textcolor{blue}{Link.}}

\item \textbf{Lee, A.} (2024). Habitat Analysis Tools [Python]. GitHub. \href{https://github.com/anzylee/habitat_analysis_public}{\textcolor{blue}{Link.}}

\item \textbf{Lee, A.} (2024). Spectral Analysis Tools for  Geomorphic Variability Functions [MATLAB]. GitHub. \href{https://github.com/anzylee/habitat_analysis_public}{\textcolor{blue}{Link.}}

\item \textbf{Lee, A.} (2020). hyporheicScalarInterFoam [C++, OpenFOAM]. HydroShare. \href{https://www.hydroshare.org/resource/dc7e13f675da42a492a4ebf1e1c337d1}{\textcolor{blue}{Link.}}


\end{rSection}


%----------------------------------------------------------------------------------------
%	TEACHING AND MENTORING
%----------------------------------------------------------------------------------------

\begin{rSection}{TEACHING \& MENTORING}
%\begin{rSubsection}{Lab Instructor and Grader}{Fall 2014}{Elementary Fluid Mechanics}{Instructor. Prof. K. D. Suh, Seoul National University}
%\item  Prepared the experimental procedures, set up the experimental apparatus, introduced the experiment, responded to student questions during the experiment, and graded student reports 
\vspace{-2.5mm}

\item {\bf Research Mentoring}, \textit{Utah State University}  \hfill {\em 2020 - Current}
\\  \textit{Undergraduate}- Maddie Witte, \textit{Graduate}- Steve White, Jared Stieve


\item {\bf Lab Instructor for Elementary Fluid Mechanics}, \textit{Purdue University}  \hfill {\em 2019}


%
%\begin{rSubsection}{Lab Instructor and Grader}{Fall 2019}{Purdue University}{Elementary Hydraulics Laboratory} \\
%\textit{Prepared the experimental procedures, set up the experimental apparatus, introduced the experiment, responded to student questions during the experiment, and graded student reports}
%

\end{rSection}


%----------------------------------------------------------------------------------------
%	REFERENCES
%----------------------------------------------------------------------------------------


%\begin{rSection}{REFERENCES}
%\vspace{-2.5mm}
%\item \textbf{Prof. Antoine Aubeneau} \hfill aubeneau@purdue.edu 
%\\ \hspace*{\fill} Lyles School of Civil engineering, Purdue University
%\vspace{2mm}
%\item \textbf{Prof. M. Bayani Cardenas} \hfill cardenas@jsg.utexas.edu
%\\ \hspace*{\fill} Jackson School of Geosciences, The University of Texas at Austin
%\vspace{2mm}
%\item \textbf{Prof. Xiaofeng Liu} \hfill xzl123@psu.edu
%\\ \hspace*{\fill} Civil and Environmental Engineering, Penn State University
%\end{rSection}
%----------------------------------------------------------------------------------------
%	EXAMPLE SECTION
%----------------------------------------------------------------------------------------

%\begin{rSection}{Section Name}

%Section content\ldots

%\end{rSection}

%----------------------------------------------------------------------------------------

\end{document}
