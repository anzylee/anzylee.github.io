%%%%%%%%%%%%%%%%%%%%%%%%%%%%%%%%%%%%%%%%%
% Medium Length Professional CV
% LaTeX Template
% Version 2.0 (8/5/13)
%
% This template has been downloaded from:
% http://www.LaTeXTemplates.com
%
% Original author:
% Trey Hunner (http://www.treyhunner.com/)
%
% Important note:
% This template requires the resume.cls file to be in the same directory as the
% .tex file. The resume.cls file provides the resume style used for structuring the
% document.
%
%%%%%%%%%%%%%%%%%%%%%%%%%%%%%%%%%%%%%%%%%

%----------------------------------------------------------------------------------------
%	PACKAGES AND OTHER DOCUMENT CONFIGURATIONS
%----------------------------------------------------------------------------------------
\pdfmapfile{=raleway.map}
\documentclass{resume_anzy} % Use the custom resume.cls style

%\usepackage[default]{opensans}
\usepackage[T1]{fontenc}
\usepackage[default]{raleway}
\usepackage{hyperref}
\usepackage{xcolor}

\usepackage[left=0.75in,top=0.6in,right=0.75in,bottom=0.6in]{geometry} % Document margins

\name{Anzy Lee} % Your name

\address{Department of Civil and Environmental Engineerig, Utah State University \\ Logan, UT 84322} % Your address
%\address{123 Pleasant Lane \\ City, State 12345} % Your secondary addess (optional)
\address{anzy.lee@usu.edu \\ \href{https://anzylee.github.io}{\textcolor{blue}{https://anzylee.github.io}}} % Your phone number and email

\begin{document}

%----------------------------------------------------------------------------------------
%	Research interests
%----------------------------------------------------------------------------------------
%\vspace{5mm}
%\begin{rSection}{Research Interests}
%\begin{rSubsection}{}{}{}{}
%\vspace{-2.5mm}
%\item {\bf Flow and transport problems:} Hydrodynamic models, CFD, Two-phase flow, Flow in porous media, Coupled systems
%\item {\bf Machine Learning:} Neural Networks, Metaheuristic Optimization Algorithm
%\end{rSubsection}
%\end{rSection}

%----------------------------------------------------------------------------------------
%	EDUCATION SECTION
%----------------------------------------------------------------------------------------

\begin{rSection}{Education}

{\bf Purdue University} \hfill {\em Aug 2016 - May 2020} \\ 
Ph.D in Civil Engineering  \\
Dissertation: Riverbed Morphology, Hydrodynamics and Hyporheic Exchange Processes \\
Advisor: Prof. Antoine Aubeneau

{\bf Seoul National University, Republic of Korea} \hfill {\em Mar 2014 - Feb 2016} \\ 
MS in Civil and Environmental Engineering   \\
Thesis: Determination of Near-global Optimal Initial Weights of Artificial Neural Network Using Harmony Search Algorithm: Application to Breakwater Armor Stones \\
Advisor: Prof. Kyung-Duck Suh 

{\bf Handong Global University, Republic of Korea } \hfill {\em Mar 2010 - Feb 2014} \\ 
BS in Spatial Environment System Engineering  
% \\
%Thesis: Derivation of the Formation Mechanism of Beach Cusp \\
%Advisor: Prof. Kyungmo Ahn
\vspace{2mm}
\end{rSection}

%----------------------------------------------------------------------------------------
%	RESEARCH EXPERIENCE SECTION
%----------------------------------------------------------------------------------------

\begin{rSection}{RESEARCH EXPERIENCE }

%------------------------------------------------

\begin{rSubsection}{Postdoctoral Scholar}{Aug 2020 - Current}{Prof. Belize Lane}{\textit{Utah State University}}
\item  Quantify the effects of geomorphological parameters on ecohydraulics and ecosystem functions
\end{rSubsection}

%------------------------------------------------
\vspace{-2.5mm}
\begin{rSubsection}{Visiting Scholar}{Aug 2020 - Current}{Prof. Greg Pasternack}{\textit{Land, Air, and Water Resources, University of California, Davis}}
\item  Develop a river archetype model representing various geomorphological features observed in natural riverine systems
\end{rSubsection}

%------------------------------------------------
\vspace{-2.5mm}
\begin{rSubsection}{Research Assistant}{Aug 2016 - Jul 2019}{Prof. Antoine Aubeneau}{\textit{Lyles School of Civil Engineering, Purdue University}}
\item Conducted numerical modeling of hyporheic exchange processes in fractal riverbed
\end{rSubsection}


%------------------------------------------------
\vspace{-2.5mm}
\begin{rSubsection}{Visiting Scholar}{Feb 2019 - Apr 2019}{Prof. Xiaofeng Liu}{\textit{Civil and Environmental Engineering, Penn State University}}
\item Developed boulder-driven hyporheic exchange model
\end{rSubsection}

%------------------------------------------------
\vspace{-2.5mm}
\begin{rSubsection}{Visiting Scholar}{Jan 2018 - Jan 2019}{Prof. M. Bayani Cardenas}{\textit{Jackson School of Geosciences, The University of Texas at Austin}}
\item  Investigated hyporheic exchange in channels with high Froude Number flows: the importance of free surface water elevation changes
\end{rSubsection}

%------------------------------------------------
%\vspace{-2.5mm}
%\begin{rSubsection}{Research Assistant}{2014 - 2015}{Prof. Kyung-Duck Suh}{\textit{Coastal Engineering Laboratory, Seoul National University}}
%\item Developed a robust hybrid Artificial Neural Network (ANN) model integrated with the Harmony search algorithm to estimate the stability number of armor unit of rubble mound structure 
%\end{rSubsection}


%------------------------------------------------

\end{rSection}

%----------------------------------------------------------------------------------------
%	COMPUTER SECTION
%----------------------------------------------------------------------------------------


%\begin{rSection}{COMPUTER SKILLS }
%\vspace{-2.5mm}
%\item \textbf{Operating Systems}: Windows, Linux
%\item \textbf{Programming}: C/C++, MATLAB, Python, MPI, Visual Basic
%\item \textbf{Scientific Applications}: \LaTeX, OpenFOAM, FEniCS, ParaView, GIS, HEC-RAS, HEC-HMS  
%\item \textbf{Technical Drawing}: Inkscape, Adobe Illustrator, AutoCAD, Microsoft Visio
%\end{rSection}
%
%\newpage
%----------------------------------------------------------------------------------------
%	TECHNICAL STRENGTHS SECTION
%----------------------------------------------------------------------------------------

\begin{rSection}{JOURNAL ARTICLES}
\vspace{-2.5mm}
\item \textbf{A. Lee}, A. Aubeneau, M. B. Cardenas, X. Liu, Boulder-driven hyporheic exchange (in preparation)
\item \textbf{A. Lee}, A. Aubeneau, M. B. Cardenas, X. Liu, Hyporheic exchange over dunes in rivers with deforming free water surface (Under review)
\item \textbf{A. Lee}, A. Aubeneau, M. B. Cardenas (2020) The Sensitivity of Hyporheic Exchange to Fractal Properties of Riverbeds. \textit{Water Resour. Res.} 56, e2019WR026560. \href{https://doi.org/10.1029/2019WR026560}{\textcolor{blue}{doi:10.1029/2019WR026560}}
\item S. W. Kim, \textbf{A. Lee}, J. Mun (2018) A Surrogate Modeling for Storm Surge Prediction Using an Artificial Neural Network. \textit{J. of Coastal Res.} 84, 866-870. \href{https://doi.org/10.2112/SI85-174.1}{\textcolor{blue}{doi:10.2112/SI85-174.1}}
\item \textbf{A. Lee}, J. W. Geem, K. D. Suh (2016) Determination of near-global optimal initial weights of artificial neural network using harmony search algorithm: Application to breakwater armor stones. \textit{Appl. Sci.} 6(6), 164. \href{https://doi.org/10.3390/app6060164}{\textcolor{blue}{doi:10.3390/app6060164}}
\item \textbf{A. Lee}, S. E. Kim, K. D. Suh (2016) An easy way to use artificial neural network model for calculating stability number of rock armor. \textit{Ocean Eng.} 127, 349-356.  \href{https://doi.org/10.1016/j.oceaneng.2016.10.013}{\textcolor{blue}{doi:10.1016/j.oceaneng.2016.10.013}}
\end{rSection}

\begin{rSection}{CONFERENCE PROCEEDINGS}
\vspace{-2.5mm}
%\begin{rSection}{CONFERENCES AND SEMINARS}
\item \textbf{A. Lee}, M. B. Cardenas, A. Aubeneau (2018) Investigation of hyporheic exchange in channels with high Froude Number flows: the importance of free surface water elevation changes, AGU 2018 Fall Meeting, Dec 2018, Washington, D.C., United States 
\item A. Aubeneau, \textbf{A. Lee}  (2018) Aris method for (reactive) transient storage models, AGU 2018 Fall Meeting, Dec 2018, Washington, D.C., United States 
\item \textbf{A. Lee}, A. Aubeneau (2017) 3D Numerical Modeling of Hyporheic Exchange Processes in Fractal Riverbed, AGU 2017 Fall Meeting, Dec 2017, New Orleans, United States 
\end{rSection}

%\newpage

\begin{rSection}{TEACHING AND MENTORING}
%\begin{rSubsection}{Lab Instructor and Grader}{Fall 2014}{Elementary Fluid Mechanics}{Instructor. Prof. K. D. Suh, Seoul National University}
%\item  Prepared the experimental procedures, set up the experimental apparatus, introduced the experiment, responded to student questions during the experiment, and graded student reports 
\begin{rSubsection}{Lab Instructor and Grader}{Fall 2019}{Elementary Hydraulics Laboratory}{Instructor. Prof. D. A. Lyn, Purdue University}
\item  Prepared the experimental procedures, set up the experimental apparatus, introduced the experiment, responded to student questions during the experiment, and graded student reports 
\end{rSubsection}
\end{rSection}

\begin{rSection}{AWARDS, SERVICE AND EXTRACURRICULAR  EXPERIENCE}
\vspace{-2.5mm}
\item \textbf{Dorothy Faye Dunn Fellowship}, \textit{Purdue University} \hfill{2019}
\item \textbf{Climate Science Summer School}, \textit{NASA JPL Center for Climate Sciences} \hfill{2018}
\item \textbf{Delleur Award}, \textit{Purdue University} \hfill{2017, 2018}
\item \textbf{Summer Institute on Earth-Surface Dynamics}, \textit{National Center for Earth-surface Dynamics} \hfill{2017}
\item \textbf{Peer Reviewer}, \textit{The journal Engineering Optimization} \hfill{2015}
%\item \textbf{LG Foundation Fellowships}, \textit{LG Yonam Foundation} \hfill{2014-2015}
%\item \textbf{Student Representative}, Spatial Environment System Engineering, \textit{Handong University} \hfill{2012}
%\item \textbf{Merit Scholarship (Top 5$\%$)}, Spatial Environment System Engineering, \textit{Handong University} \hfill{2012}
\end{rSection}



%----------------------------------------------------------------------------------------
%	REFERENCES
%----------------------------------------------------------------------------------------


%\begin{rSection}{REFERENCES}
%\vspace{-2.5mm}
%\item \textbf{Prof. Antoine Aubeneau} \hfill aubeneau@purdue.edu 
%\\ \hspace*{\fill} Lyles School of Civil engineering, Purdue University
%\vspace{2mm}
%\item \textbf{Prof. M. Bayani Cardenas} \hfill cardenas@jsg.utexas.edu
%\\ \hspace*{\fill} Jackson School of Geosciences, The University of Texas at Austin
%\vspace{2mm}
%\item \textbf{Prof. Xiaofeng Liu} \hfill xzl123@psu.edu
%\\ \hspace*{\fill} Civil and Environmental Engineering, Penn State University
%\end{rSection}
%----------------------------------------------------------------------------------------
%	EXAMPLE SECTION
%----------------------------------------------------------------------------------------

%\begin{rSection}{Section Name}

%Section content\ldots

%\end{rSection}

%----------------------------------------------------------------------------------------

\end{document}
